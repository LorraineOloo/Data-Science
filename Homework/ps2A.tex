% Options for packages loaded elsewhere
\PassOptionsToPackage{unicode}{hyperref}
\PassOptionsToPackage{hyphens}{url}
%
\documentclass[
]{article}
\usepackage{amsmath,amssymb}
\usepackage{lmodern}
\usepackage{ifxetex,ifluatex}
\ifnum 0\ifxetex 1\fi\ifluatex 1\fi=0 % if pdftex
  \usepackage[T1]{fontenc}
  \usepackage[utf8]{inputenc}
  \usepackage{textcomp} % provide euro and other symbols
\else % if luatex or xetex
  \usepackage{unicode-math}
  \defaultfontfeatures{Scale=MatchLowercase}
  \defaultfontfeatures[\rmfamily]{Ligatures=TeX,Scale=1}
\fi
% Use upquote if available, for straight quotes in verbatim environments
\IfFileExists{upquote.sty}{\usepackage{upquote}}{}
\IfFileExists{microtype.sty}{% use microtype if available
  \usepackage[]{microtype}
  \UseMicrotypeSet[protrusion]{basicmath} % disable protrusion for tt fonts
}{}
\makeatletter
\@ifundefined{KOMAClassName}{% if non-KOMA class
  \IfFileExists{parskip.sty}{%
    \usepackage{parskip}
  }{% else
    \setlength{\parindent}{0pt}
    \setlength{\parskip}{6pt plus 2pt minus 1pt}}
}{% if KOMA class
  \KOMAoptions{parskip=half}}
\makeatother
\usepackage{xcolor}
\IfFileExists{xurl.sty}{\usepackage{xurl}}{} % add URL line breaks if available
\IfFileExists{bookmark.sty}{\usepackage{bookmark}}{\usepackage{hyperref}}
\hypersetup{
  pdftitle={STAT 231: Problem Set 2A},
  pdfauthor={Lorraine Oloo},
  hidelinks,
  pdfcreator={LaTeX via pandoc}}
\urlstyle{same} % disable monospaced font for URLs
\usepackage[margin=1in]{geometry}
\usepackage{color}
\usepackage{fancyvrb}
\newcommand{\VerbBar}{|}
\newcommand{\VERB}{\Verb[commandchars=\\\{\}]}
\DefineVerbatimEnvironment{Highlighting}{Verbatim}{commandchars=\\\{\}}
% Add ',fontsize=\small' for more characters per line
\usepackage{framed}
\definecolor{shadecolor}{RGB}{248,248,248}
\newenvironment{Shaded}{\begin{snugshade}}{\end{snugshade}}
\newcommand{\AlertTok}[1]{\textcolor[rgb]{0.94,0.16,0.16}{#1}}
\newcommand{\AnnotationTok}[1]{\textcolor[rgb]{0.56,0.35,0.01}{\textbf{\textit{#1}}}}
\newcommand{\AttributeTok}[1]{\textcolor[rgb]{0.77,0.63,0.00}{#1}}
\newcommand{\BaseNTok}[1]{\textcolor[rgb]{0.00,0.00,0.81}{#1}}
\newcommand{\BuiltInTok}[1]{#1}
\newcommand{\CharTok}[1]{\textcolor[rgb]{0.31,0.60,0.02}{#1}}
\newcommand{\CommentTok}[1]{\textcolor[rgb]{0.56,0.35,0.01}{\textit{#1}}}
\newcommand{\CommentVarTok}[1]{\textcolor[rgb]{0.56,0.35,0.01}{\textbf{\textit{#1}}}}
\newcommand{\ConstantTok}[1]{\textcolor[rgb]{0.00,0.00,0.00}{#1}}
\newcommand{\ControlFlowTok}[1]{\textcolor[rgb]{0.13,0.29,0.53}{\textbf{#1}}}
\newcommand{\DataTypeTok}[1]{\textcolor[rgb]{0.13,0.29,0.53}{#1}}
\newcommand{\DecValTok}[1]{\textcolor[rgb]{0.00,0.00,0.81}{#1}}
\newcommand{\DocumentationTok}[1]{\textcolor[rgb]{0.56,0.35,0.01}{\textbf{\textit{#1}}}}
\newcommand{\ErrorTok}[1]{\textcolor[rgb]{0.64,0.00,0.00}{\textbf{#1}}}
\newcommand{\ExtensionTok}[1]{#1}
\newcommand{\FloatTok}[1]{\textcolor[rgb]{0.00,0.00,0.81}{#1}}
\newcommand{\FunctionTok}[1]{\textcolor[rgb]{0.00,0.00,0.00}{#1}}
\newcommand{\ImportTok}[1]{#1}
\newcommand{\InformationTok}[1]{\textcolor[rgb]{0.56,0.35,0.01}{\textbf{\textit{#1}}}}
\newcommand{\KeywordTok}[1]{\textcolor[rgb]{0.13,0.29,0.53}{\textbf{#1}}}
\newcommand{\NormalTok}[1]{#1}
\newcommand{\OperatorTok}[1]{\textcolor[rgb]{0.81,0.36,0.00}{\textbf{#1}}}
\newcommand{\OtherTok}[1]{\textcolor[rgb]{0.56,0.35,0.01}{#1}}
\newcommand{\PreprocessorTok}[1]{\textcolor[rgb]{0.56,0.35,0.01}{\textit{#1}}}
\newcommand{\RegionMarkerTok}[1]{#1}
\newcommand{\SpecialCharTok}[1]{\textcolor[rgb]{0.00,0.00,0.00}{#1}}
\newcommand{\SpecialStringTok}[1]{\textcolor[rgb]{0.31,0.60,0.02}{#1}}
\newcommand{\StringTok}[1]{\textcolor[rgb]{0.31,0.60,0.02}{#1}}
\newcommand{\VariableTok}[1]{\textcolor[rgb]{0.00,0.00,0.00}{#1}}
\newcommand{\VerbatimStringTok}[1]{\textcolor[rgb]{0.31,0.60,0.02}{#1}}
\newcommand{\WarningTok}[1]{\textcolor[rgb]{0.56,0.35,0.01}{\textbf{\textit{#1}}}}
\usepackage{graphicx}
\makeatletter
\def\maxwidth{\ifdim\Gin@nat@width>\linewidth\linewidth\else\Gin@nat@width\fi}
\def\maxheight{\ifdim\Gin@nat@height>\textheight\textheight\else\Gin@nat@height\fi}
\makeatother
% Scale images if necessary, so that they will not overflow the page
% margins by default, and it is still possible to overwrite the defaults
% using explicit options in \includegraphics[width, height, ...]{}
\setkeys{Gin}{width=\maxwidth,height=\maxheight,keepaspectratio}
% Set default figure placement to htbp
\makeatletter
\def\fps@figure{htbp}
\makeatother
\setlength{\emergencystretch}{3em} % prevent overfull lines
\providecommand{\tightlist}{%
  \setlength{\itemsep}{0pt}\setlength{\parskip}{0pt}}
\setcounter{secnumdepth}{-\maxdimen} % remove section numbering
\ifluatex
  \usepackage{selnolig}  % disable illegal ligatures
\fi

\title{STAT 231: Problem Set 2A}
\author{Lorraine Oloo}
\date{due by 5 PM on Monday, March 1}

\begin{document}
\maketitle

In order to most effectively digest the textbook chapter readings -- and
the new R commands each presents -- series A homework assignments are
designed to encourage you to read the textbook chapters actively and in
line with the textbook's Prop Tip of page 33:

``\textbf{Pro Tip}: If you want to learn how to use a particular
command, we highly recommend running the example code on your own''

A more thorough reading and light practice of the textbook chapter prior
to class allows us to dive quicker and deeper into the topics and
commands during class. Furthermore, learning a programming lanugage is
like learning any other language -- practice, practice, practice is the
key to fluency. By having two assignments each week, I hope to encourage
practice throughout the week. A little coding each day will take you a
long way!

\emph{Series A assignments are intended to be completed individually.}
While most of our work in this class will be collaborative, it is
important each individual completes the active readings. The problems
should be straightforward based on the textbook readings, but if you
have any questions, feel free to ask me!

Steps to proceed:

\begin{enumerate}
\item In RStudio, go to File > Open Project, navigate to the folder with the course-content repo, select the course-content project (course-content.Rproj), and click "Open" 
\item Pull the course-content repo (e.g. using the blue-ish down arrow in the Git tab in upper right window)
\item Copy ps2A.Rmd from the course repo to your repo (see page 6 of the GitHub Classroom Guide for Stat231 if needed)
\item Close the course-content repo project in RStudio
\item Open YOUR repo project in RStudio
\item In the ps2A.Rmd file in YOUR repo, replace "YOUR NAME HERE" with your name
\item Add in your responses, committing and pushing to YOUR repo in appropriate places along the way
\item Run "Knit PDF" 
\item Upload the pdf to Gradescope.  Don't forget to select which of your pages are associated with each problem.  \textit{You will not get credit for work on unassigned pages (e.g., if you only selected the first page but your solution spans two pages, you would lose points for any part on the second page that the grader can't see).} 
\end{enumerate}

\newpage

\hypertarget{nyc-flights}{%
\section{1. NYC Flights}\label{nyc-flights}}

\hypertarget{a.}{%
\subsubsection{a.}\label{a.}}

In Section 4.3.1, the \texttt{flights} and \texttt{carrier} tables
within the \texttt{nycflights13} package are joined together. Recreate
the \texttt{flightsJoined} dataset from page 80. Hint: make sure you've
loaded the \texttt{nycflights13} package before referring to the data
tables (see code on page 79).

\begin{Shaded}
\begin{Highlighting}[]
\FunctionTok{library}\NormalTok{(nycflights13)}
\CommentTok{\#glimpse(flights)}

\NormalTok{flights\_joined }\OtherTok{\textless{}{-}}\NormalTok{ flights }\SpecialCharTok{\%\textgreater{}\%} 
  \FunctionTok{inner\_join}\NormalTok{(airlines, }\AttributeTok{by =} \FunctionTok{c}\NormalTok{(}\StringTok{"carrier"} \OtherTok{=} \StringTok{"carrier"}\NormalTok{))}
\FunctionTok{glimpse}\NormalTok{(flights\_joined)}
\end{Highlighting}
\end{Shaded}

\begin{verbatim}
## Rows: 336,776
## Columns: 20
## $ year           <int> 2013, 2013, 2013, 2013, 2013, 2013, 2013, 2013, 2013, 2~
## $ month          <int> 1, 1, 1, 1, 1, 1, 1, 1, 1, 1, 1, 1, 1, 1, 1, 1, 1, 1, 1~
## $ day            <int> 1, 1, 1, 1, 1, 1, 1, 1, 1, 1, 1, 1, 1, 1, 1, 1, 1, 1, 1~
## $ dep_time       <int> 517, 533, 542, 544, 554, 554, 555, 557, 557, 558, 558, ~
## $ sched_dep_time <int> 515, 529, 540, 545, 600, 558, 600, 600, 600, 600, 600, ~
## $ dep_delay      <dbl> 2, 4, 2, -1, -6, -4, -5, -3, -3, -2, -2, -2, -2, -2, -1~
## $ arr_time       <int> 830, 850, 923, 1004, 812, 740, 913, 709, 838, 753, 849,~
## $ sched_arr_time <int> 819, 830, 850, 1022, 837, 728, 854, 723, 846, 745, 851,~
## $ arr_delay      <dbl> 11, 20, 33, -18, -25, 12, 19, -14, -8, 8, -2, -3, 7, -1~
## $ carrier        <chr> "UA", "UA", "AA", "B6", "DL", "UA", "B6", "EV", "B6", "~
## $ flight         <int> 1545, 1714, 1141, 725, 461, 1696, 507, 5708, 79, 301, 4~
## $ tailnum        <chr> "N14228", "N24211", "N619AA", "N804JB", "N668DN", "N394~
## $ origin         <chr> "EWR", "LGA", "JFK", "JFK", "LGA", "EWR", "EWR", "LGA",~
## $ dest           <chr> "IAH", "IAH", "MIA", "BQN", "ATL", "ORD", "FLL", "IAD",~
## $ air_time       <dbl> 227, 227, 160, 183, 116, 150, 158, 53, 140, 138, 149, 1~
## $ distance       <dbl> 1400, 1416, 1089, 1576, 762, 719, 1065, 229, 944, 733, ~
## $ hour           <dbl> 5, 5, 5, 5, 6, 5, 6, 6, 6, 6, 6, 6, 6, 6, 6, 5, 6, 6, 6~
## $ minute         <dbl> 15, 29, 40, 45, 0, 58, 0, 0, 0, 0, 0, 0, 0, 0, 0, 59, 0~
## $ time_hour      <dttm> 2013-01-01 05:00:00, 2013-01-01 05:00:00, 2013-01-01 0~
## $ name           <chr> "United Air Lines Inc.", "United Air Lines Inc.", "Amer~
\end{verbatim}

\hypertarget{b.}{%
\subsubsection{b.}\label{b.}}

Now, create a new dataset \texttt{flightsJoined2} that:

\begin{itemize}
\tightlist
\item
  creates a new variable, \texttt{distance\_km}, which is distance in
  kilometers (note that 1 mile is about 1.6 kilometers)
\item
  keeps only the variables: \texttt{name}, \texttt{flight},
  \texttt{arr\_delay}, and \texttt{distance\_km}\\
\item
  keeps only observations where distance is less than 500 kilometers
\end{itemize}

Hint: see examples in Section 4.1 for subsetting datasets and creating
new variables.

\begin{Shaded}
\begin{Highlighting}[]
\FunctionTok{select}\NormalTok{(flights\_joined2 }\OtherTok{\textless{}{-}}\NormalTok{ flights\_joined }\SpecialCharTok{\%\textgreater{}\%} 
  
  \FunctionTok{mutate}\NormalTok{(}\AttributeTok{distance\_km =}\NormalTok{ distance}\SpecialCharTok{*}\FloatTok{1.6}\NormalTok{),name, flight, arr\_delay,distance\_km) }
\end{Highlighting}
\end{Shaded}

\begin{verbatim}
## # A tibble: 336,776 x 4
##    name                     flight arr_delay distance_km
##    <chr>                     <int>     <dbl>       <dbl>
##  1 United Air Lines Inc.      1545        11       2240 
##  2 United Air Lines Inc.      1714        20       2266.
##  3 American Airlines Inc.     1141        33       1742.
##  4 JetBlue Airways             725       -18       2522.
##  5 Delta Air Lines Inc.        461       -25       1219.
##  6 United Air Lines Inc.      1696        12       1150.
##  7 JetBlue Airways             507        19       1704 
##  8 ExpressJet Airlines Inc.   5708       -14        366.
##  9 JetBlue Airways              79        -8       1510.
## 10 American Airlines Inc.      301         8       1173.
## # ... with 336,766 more rows
\end{verbatim}

\hypertarget{c.}{%
\subsubsection{c.}\label{c.}}

Lastly, using the functions introduced in Section 4.1.4, compute the
number of flights (call this \texttt{N}), the average arrival delay
(call this \texttt{avg\_arr\_delay}), and the average distance in
kilometers (call this \texttt{avg\_dist\_km}) among these flights with
distances less than 500 km (i.e.~working off of \texttt{flightsJoined2})
\emph{grouping by the carrier name}. Sort the results in descending
order based on \texttt{avg\_arr\_delay}.

Getting NAs for \texttt{avg\_arr\_delay}? That happens when some
observations are missing that data. Before grouping and summarizing, add
a line to exclude observations with missing arrival delay information
using \texttt{filter(is.na(arr\_delay)==FALSE)}.

\begin{Shaded}
\begin{Highlighting}[]
\NormalTok{flights\_joined3}\OtherTok{\textless{}{-}}\NormalTok{ flights\_joined2 }\SpecialCharTok{\%\textgreater{}\%}
  \FunctionTok{select}\NormalTok{(name, flight,arr\_delay,distance\_km) }\SpecialCharTok{\%\textgreater{}\%}
  \FunctionTok{filter}\NormalTok{(distance\_km}\SpecialCharTok{\textless{}}\DecValTok{500} \SpecialCharTok{\&} \FunctionTok{is.na}\NormalTok{(arr\_delay)}\SpecialCharTok{==}\ConstantTok{FALSE}\NormalTok{)}

\NormalTok{fj}\OtherTok{\textless{}{-}}\NormalTok{ flights\_joined3 }\SpecialCharTok{\%\textgreater{}\%} 
  \FunctionTok{group\_by}\NormalTok{(name) }\SpecialCharTok{\%\textgreater{}\%} 
  \FunctionTok{summarize}\NormalTok{(}
    \AttributeTok{N =} \FunctionTok{sum}\NormalTok{(flight),}
    \AttributeTok{avg\_arr\_delay =} \FunctionTok{mean}\NormalTok{(arr\_delay),}
    \AttributeTok{avg\_dist\_km =} \FunctionTok{mean}\NormalTok{(distance\_km)}
\NormalTok{  )}

\FunctionTok{arrange}\NormalTok{(fj,}\FunctionTok{desc}\NormalTok{(avg\_arr\_delay))}
\end{Highlighting}
\end{Shaded}

\begin{verbatim}
## # A tibble: 11 x 4
##    name                            N avg_arr_delay avg_dist_km
##    <chr>                       <int>         <dbl>       <dbl>
##  1 Mesa Airlines Inc.        1071507        18.0          360.
##  2 ExpressJet Airlines Inc. 70431260        15.6          373.
##  3 Envoy Air                10207234        11.0          351.
##  4 JetBlue Airways          12555455         8.66         385.
##  5 Endeavor Air Inc.        22179923         6.82         339.
##  6 Southwest Airlines Co.     289201         4.92         272.
##  7 United Air Lines Inc.     3164913         4.09         320.
##  8 SkyWest Airlines Inc.        4967         3            366.
##  9 US Airways Inc.          19331098         2.22         308.
## 10 American Airlines Inc.    1999580         1.88         299.
## 11 Delta Air Lines Inc.      1679867        -0.643        325.
\end{verbatim}

\newpage

\hypertarget{baby-names}{%
\section{2. Baby names}\label{baby-names}}

\hypertarget{a.-1}{%
\subsubsection{a.}\label{a.-1}}

Working with the \texttt{babynames} data table in the \texttt{babynames}
package, create a dataset \texttt{babynames2} that only includes years
2000 to 2017.

\begin{Shaded}
\begin{Highlighting}[]
\NormalTok{data }\OtherTok{=}\NormalTok{ babynames}\SpecialCharTok{::}\NormalTok{babynames}

\NormalTok{babynames2}\OtherTok{\textless{}{-}}\NormalTok{ babynames}\SpecialCharTok{::}\NormalTok{babynames }\SpecialCharTok{\%\textgreater{}\%}
  
  \FunctionTok{filter}\NormalTok{(year }\SpecialCharTok{\%in\%} \DecValTok{2000}\SpecialCharTok{:}\DecValTok{2017}\NormalTok{)}

\NormalTok{babynames2  }
\end{Highlighting}
\end{Shaded}

\begin{verbatim}
## # A tibble: 591,925 x 5
##     year sex   name          n    prop
##    <dbl> <chr> <chr>     <int>   <dbl>
##  1  2000 F     Emily     25953 0.0130 
##  2  2000 F     Hannah    23080 0.0116 
##  3  2000 F     Madison   19967 0.0100 
##  4  2000 F     Ashley    17997 0.00902
##  5  2000 F     Sarah     17697 0.00887
##  6  2000 F     Alexis    17629 0.00884
##  7  2000 F     Samantha  17266 0.00866
##  8  2000 F     Jessica   15709 0.00787
##  9  2000 F     Elizabeth 15094 0.00757
## 10  2000 F     Taylor    15078 0.00756
## # ... with 591,915 more rows
\end{verbatim}

\hypertarget{b.-1}{%
\subsubsection{b.}\label{b.-1}}

Following the code presented in Section 5.2.4, create a dataset called
\texttt{BabyNarrow} that summarizes the total number of people with each
name (born between 2000 and 2017), grouped by sex. (Hint: follow the
second code chunk on page 102, but don't filter on any particular
names.) Look at the dataset. Why have we called this dataset ``narrow''?

\begin{quote}
ANSWER: It has 2 fewer columns compared to the previous data set.
\end{quote}

\begin{Shaded}
\begin{Highlighting}[]
\NormalTok{BabyNarrow}\OtherTok{\textless{}{-}}\NormalTok{ babynames2 }\SpecialCharTok{\%\textgreater{}\%}
\FunctionTok{group\_by}\NormalTok{(name, sex) }\SpecialCharTok{\%\textgreater{}\%}
\FunctionTok{summarise}\NormalTok{(}\AttributeTok{total =} \FunctionTok{sum}\NormalTok{(n))}
\end{Highlighting}
\end{Shaded}

\begin{verbatim}
## `summarise()` has grouped output by 'name'. You can override using the `.groups` argument.
\end{verbatim}

\begin{Shaded}
\begin{Highlighting}[]
\NormalTok{BabyNarrow}
\end{Highlighting}
\end{Shaded}

\begin{verbatim}
## # A tibble: 73,332 x 3
## # Groups:   name [67,063]
##    name      sex   total
##    <chr>     <chr> <int>
##  1 Aaban     M       107
##  2 Aabha     F        35
##  3 Aabid     M        10
##  4 Aabir     M         5
##  5 Aabriella F        32
##  6 Aada      F         5
##  7 Aadam     M       202
##  8 Aadan     M       130
##  9 Aadarsh   M       199
## 10 Aaden     F         5
## # ... with 73,322 more rows
\end{verbatim}

\hypertarget{c.-1}{%
\subsubsection{c.}\label{c.-1}}

Now, following the code chunk presented on page 103*, put the data into
a wide format (call the new dataset \texttt{BabyWide}), and only keep
observations where both \texttt{M} and \texttt{F} are greater than
10,000. Compute the \texttt{ratio} (as \texttt{pmin(M/F,\ F/M})) and
identify the top three names with the largest ratio. (Note: these names
could be different from the ones found on page 103 since we limited the
dataset to years 2000-2017 and names with greater than 10,000
individuals.)

\begin{itemize}
\tightlist
\item
  Note: you can use the \texttt{pivot\_wider()} function instead of the
  \texttt{spread()} function if using the 2nd edition of the textbook
  (e.g., see
  \href{https://mdsr-book.github.io/mdsr2e/ch-dataII.html\#pivoting-wider}{Section
  6.2.2} and
  \href{https://mdsr-book.github.io/mdsr2e/ch-dataII.html\#pivoting-longer}{6.2.3}
  in the 2nd edition). I find \texttt{pivot\_wider()} and
  \texttt{pivot\_longer()} to be more intuitive than \texttt{spread()}
  and \texttt{gather()}.
\end{itemize}

\begin{quote}
ANSWER:
\end{quote}

\begin{Shaded}
\begin{Highlighting}[]
\CommentTok{\# this will bring up "Pivoting Introduction" vignette in your Help tab}
\CommentTok{\#vignette("pivot")}
\NormalTok{BabyWide }\OtherTok{\textless{}{-}}\NormalTok{ babynames}\SpecialCharTok{::}\NormalTok{babynames }\SpecialCharTok{\%\textgreater{}\%}
\FunctionTok{group\_by}\NormalTok{(sex, name) }\SpecialCharTok{\%\textgreater{}\%}
\FunctionTok{summarize}\NormalTok{(}\AttributeTok{total =} \FunctionTok{sum}\NormalTok{(n)) }\SpecialCharTok{\%\textgreater{}\%}
\FunctionTok{spread}\NormalTok{(}\AttributeTok{key =}\NormalTok{ sex, }\AttributeTok{value =}\NormalTok{ total, }\AttributeTok{fill =} \DecValTok{0}\NormalTok{)}
\end{Highlighting}
\end{Shaded}

\begin{verbatim}
## `summarise()` has grouped output by 'sex'. You can override using the `.groups` argument.
\end{verbatim}

\begin{Shaded}
\begin{Highlighting}[]
\FunctionTok{head}\NormalTok{(BabyWide, }\DecValTok{3}\NormalTok{)}
\end{Highlighting}
\end{Shaded}

\begin{verbatim}
## # A tibble: 3 x 3
##   name      F     M
##   <chr> <dbl> <dbl>
## 1 Aaban     0   107
## 2 Aabha    35     0
## 3 Aabid     0    10
\end{verbatim}

\hypertarget{d.}{%
\subsubsection{d.}\label{d.}}

Lastly, use the \texttt{gather()} function (or the
\texttt{pivot\_longer()} function) to put the dataset back into narrow
form. Call this dataset \texttt{BabyNarrow2}. Hint: see Section 5.2.3.
Why are the number of observations in \texttt{BabyNarrow2} different
from that in \texttt{BabyNarrow}?

\begin{quote}
ANSWER:
\end{quote}

\end{document}
